\documentclass[a4paper, titlepage]{article}
\usepackage[url = false, sorting = none]{biblatex}
\usepackage{helvet}
\addbibresource{library.bib}
\DeclareNameAlias{author}{family-given}



\begin{document}

\title{Paper title}
\author{
	Diego Lopes\\
	\and
	Moisés Teles dos Santos}
\date{01/07/2025}
\maketitle

\begin{abstract}
	This is the papers abstract, version 2!
\end{abstract}

\section{Introduction}

Since 1961, worldwide food supply per capita has raised by 30\%, in large part because of the increase of use of
nitrogen fertilizers to improve agricultural productivity \cite{mbowIPCCSpecialReport2019}. The importance of
fertilizers on agriculture cannot be overstated, with research indicating that half of the world’s population
is sustained by mineral fertilizers. The agricultural sector is also responsible for between 11\% and 15\% of
all greenhouse gas (GHG) emissions worldwide \cite{ifaEstimatingReportingFertilizerRelated2016}, with fertilizers
accounting for 2\% to 3\% of emissions \cite{brentrupCarbonFootprintAnalysis2016}. Production of mineral fertilizers
is dependent on fossil fuels and raw materials, including mineral extraction,
transportation, manufacturing, and power supply. In this scenario, decarbonizing the fertilizer sector is equally
important to the industry’s expansion \cite{ouikhalfanNetZeroEmissionFertilizers2022}, considering the GHG emissions
reduction preconized in the Paris agreement (2015).

The mineral fertilizers divide themselves into three main categories, each one corresponding to the main
macronutrient present in its composition: nitrogen (N), phosphorus (P) and potassium (K). Although none of the
macronutrients can be considered more or less important than the other, nitrogen is the one consumed at higher
volumes, and also the one with the more energetically intensive manufacturing
process \cite{ieaAmmoniaTechnologyRoadmap2021}. Out of all nitrogen fertilizers, urea is the main commercial product,
with ammonia as its obligatory precursor. Ammonia ($NH_3$) is obtained by the synthesis of hydrogen ($H_2$)
and nitrogen $N_2$ through the traditional Haber-Bosch process, responsible for the production of more than 90\% of
worldwide ammonia \cite{applAmmoniaPrinciplesIndustrial1999}. Nitrogen is obtained from the atmosphere,
but hydrogen is traditionally obtained from steam reforming of fossil fuels; 72\% of the worldwide ammonia
production comes from natural gas reforming, while 26\% is obtained from coal gasification. 1\% is produced
through other petroleum derivatives, while the renewable fraction, produced through water electrolysis,
corresponds to less than 1\% \cite{ieaAmmoniaTechnologyRoadmap2021}. Urea, in turn, is produced by the Basarov reaction
using ammonia and carbon dioxide ($CO_2$), with this process being responsible for all commercial urea production
in a large scale \cite{meessenUrea2010}. The $CO_2$ is supplied by the reforming syngas, as both $CO$ and $CO_2$ must
be removed prior to the ammonia synthesis as they are poisons to the commercial catalysts.

Since both processes are energetically intensive, efforts were made throughout the 20th century to improve its energy
efficiency, involving equipment changes, process control and residual heat utilization. With these efforts, current
ammonia production is very close to the theoretical minimum energy consumption \cite{ieaAmmoniaTechnologyRoadmap2021}.
Considering the small efficiency improvements still possible, a natural future step is to search for sustainable
replacements for the process’ feedstocks and energy usages.

In this scenario, biomass is a promising feedstock replacement. Among the conversion routes, gasification is of special
interest considering that the biomass syngas composition is similar to the fossil fuel syngas obtained in traditional
processes; in that sense the conversion technologies of this syngas into alcohols, hydrocarbons and other chemicals
can be leveraged with significant overlap.





\printbibliography{}


\end{document}
