\documentclass[a4paper, titlepage]{article}
\usepackage{biblatex}
\addbibresource{library.bib}



\begin{document}

\title{Paper title}
\author{
	Diego Lopes\\
	\and
	Moises Teles dos Santos}
\date{11/03/2025}
\maketitle

\begin{abstract}
	This is the papers abstract, version 2!
\end{abstract}

\section{Introduction}
Since 1961, worldwide food supply per capita has raised by 30\%, in large part because of the increase of use of
nitrogen fertilizers to improve agricultural productivity [1]. The importance of fertilizers on agriculture cannot
be overstated, with research indicating that half of the world’s population is sustained by mineral fertilizers
The agricultural sector is also responsible for between 11\% and 15\% of all greenhouse gas (GHG)
emissions worldwide [2], with fertilizers accounting for 2\% to 3\% of emissions [3]. Production of mineral
fertilizers is dependent on fossil fuels and raw materials, including mineral extraction,
transportation, manufacturing, and power supply. In this scenario, decarbonizing the fertilizer sector is equally
important to the industry’s expansion [4], considering the GHG emissions reduction preconized in
the Paris agreement (2015).




\printbibliography{}
\end{document}
