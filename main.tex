\documentclass[a4paper, titlepage]{article}
\renewcommand{\baselinestretch}{1.15}				% line spacing 
\usepackage[url=false, 
			sorting=none, 
			maxcitenames=2,
			hyperref=true]{biblatex}	% remove url visit date and sort references by appearance
\usepackage{hyperref}
\usepackage{helvet}									% helvetica font
\usepackage{array}
\addbibresource{library.bib}
\DeclareNameAlias{author}{family-given}

% ideas for paper
% add carbon taxes
% add ports as possible sinks
% add eucalyptus
% add electrolysis


\begin{document}
\title{Paper title}
\author{
	Diego Lopes\\
	\and
	Moisés Teles dos Santos}
\date{01/07/2025}
\maketitle

\begin{abstract}
	This is the papers abstract, version 2!
\end{abstract}

\section{Introduction}

Since 1961, worldwide food supply per capita has raised by 30\%, in large part because of the increase of use of
nitrogen fertilizers to improve agricultural productivity \cite{mbowIPCCSpecialReport2019}. The importance of
fertilizers on agriculture cannot be overstated, with research indicating that half of the world’s population
is sustained by mineral fertilizers. The agricultural sector is also responsible for between 11\% and 15\% of
all greenhouse gas (GHG) emissions worldwide \cite{ifaEstimatingReportingFertilizerRelated2016}, with fertilizers
accounting for 2\% to 3\% of emissions \cite{brentrupCarbonFootprintAnalysis2016}. Production of mineral fertilizers
is dependent on fossil fuels and raw materials, including mineral extraction,
transportation, manufacturing, and power supply. In this scenario, decarbonizing the fertilizer sector is equally
important to the industry’s expansion \cite{ouikhalfanNetZeroEmissionFertilizers2022}, considering the GHG emissions
reduction preconized in the Paris agreement (2015).

The mineral fertilizers divide themselves into three main categories, each one corresponding to the main
macronutrient present in its composition: nitrogen (N), phosphorus (P) and potassium (K). Although none of the
macronutrients can be considered more or less important than the other, nitrogen is the one consumed at higher
volumes, and also the one with the more energetically intensive manufacturing
process \cite{ieaAmmoniaTechnologyRoadmap2021}. Out of all nitrogen fertilizers, urea is the main commercial product,
with ammonia as its obligatory precursor. Ammonia ($NH_3$) is obtained by the synthesis of hydrogen ($H_2$)
and nitrogen $N_2$ through the traditional Haber-Bosch process, responsible for the production of more than 90\% of
worldwide ammonia \cite{applAmmoniaPrinciplesIndustrial1999}. Nitrogen is obtained from the atmosphere,
but hydrogen is traditionally obtained from steam reforming of fossil fuels; 72\% of the worldwide ammonia
production comes from natural gas reforming, while 26\% is obtained from coal gasification. 1\% is produced
through other petroleum derivatives, while the renewable fraction, produced through water electrolysis,
corresponds to less than 1\% \cite{ieaAmmoniaTechnologyRoadmap2021}. Urea, in turn, is produced by the Basarov reaction
using ammonia and carbon dioxide ($CO_2$), with this process being responsible for all commercial urea production
in a large scale \cite{meessenUrea2010}. The $CO_2$ is supplied by the reforming syngas, as both $CO$ and $CO_2$ must
be removed prior to the ammonia synthesis as they are poisons to the commercial catalysts.

Since both processes are energetically intensive, efforts were made throughout the 20\textsuperscript{th} century to improve its energy
efficiency, involving equipment changes, process control and residual heat utilization. With these efforts, current
ammonia production is very close to the theoretical minimum energy consumption \cite{ieaAmmoniaTechnologyRoadmap2021}.
Considering the small efficiency improvements still possible, a natural future step is to search for sustainable
replacements for the process’ feedstocks and energy usages.

In this scenario, biomass is a promising feedstock replacement. Among the conversion routes, gasification is of special
interest considering that the biomass syngas composition is similar to the fossil fuel syngas obtained in traditional
processes; in that sense the conversion technologies of this syngas into alcohols, hydrocarbons and other chemicals
can be leveraged with significant overlap. Despite that, biorefineries pose additional challenges to the conventional
plants. Given its high volume and low energetic density compared to fossil fuels, the feasibility of all biorefineries
depend heavily on the local availability of biomass inputs and spatial allocation of this biomass
. This demonstrates the need for an integrated approach in biorefinery
modeling that combines technical performance, location, capacity and configuration of the plant \cite{schroderImprovingBiorefineryPlanning2018}.

\section{Literature Review}

The mathematical modeling of biomass gasification, ammonia, urea and its subprocesses has been the object of multiple
studies. \textcite{baruahModelingBiomassGasification2014} presented a systematic review of  existing techniques to model
biomass gasification systems, categorized by gasifier type, feedstock and parameters studied,
concluding that equilibrium modeling is the most suitable technique for process studies on the influence of fuel and
process parameters, since these models are independent of gasifier's design.
\textcite{gambarottaNonstoichiometricEquilibriumModel2018} and \textcite{azzoneDevelopmentEquilibriumModel2012} developed
equilibrium models for biomass gasification and compared the model’s results against experimental data, predicting the
syngas composition with reasonable accuracy. Gasification models were successfully implemented in commercial process
simulators such as Aspen Plus, in \textcite{hanModelingDowndraftBiomass2017} and
\textcite{ramzanSimulationHybridBiomass2011}

The ammonia synthesis process has also been modeled extensively. The reaction mechanisms and kinetics have been
studied since the 1930s \cite{gillespieThermodynamicTreatmentChemical1930} and the Temkin-Pyzhev formulation is
used to this day in ammonia reactor designs \cite{temkinKineticsAmmoniaSynthesis1940}
\cite{singhSimulationAmmoniaSynthesis1979} \cite{nielsenAmmoniaCatalysisManufacture1995}.
\textcite{florez-orregoProcessSynthesisOptimization2018} has modeled the full fossil fuel process and conducted
exergy analysis and optimization while \textcite{domingosExergyEnvironmentalAnalysis2021} has applied the same model
for ammonia production through black liquor gasification, with promising results reaching negative emissions in
certain scenarios. Ammonia production via other biomasses have also been proposed by
\textcite{florez-orregoComparativeExergyEconomic2019} and \textcite{tunaTechnoeconomicAssessmentNonfossil2014}, with
both studies identifying substantial environmental benefits in using biomass as a feedstock, despite a lower energy
efficiency in the process.

The urea process is also studied since the start of the 20\textsuperscript{th} century, with early works on the
urea synthesis published by \textcite{frejacquesBasesTheoriquesSynthese1948} and
\textcite{kawasumiEquilibriumCO2NH3UreaH2OSystem1953}, and \textcite{islaSimulationUreaSynthesis1993}, all focusing
on describing and modeling the complex thermodynamics in the urea reactor. Despite this, more extensive models of
the full process are far rarer, as are attempts at optimizing it.
\textcite{meessenUreaSynthesis2014} describes the main commercial processes available in the industry, and
\textcite{aspentechASPEN88Technical2011} published a process model for the urea synthesis loop based on the
Stamicarbon stripping process. Given the difficulties in predicting the urea $NH_3$-$CO_2$-water equilibrium, this
model has been leveraged directly by several authors.

\textcite{zhangTechnoeconomicComparison1002021} presented process models for the full gasification, ammonia and urea
process with the traditional fossil fuel route and two renewable alternatives, using biomass gasification and water
electrolysis. \textcite{alfianMultiobjectiveOptimizationGreen2019} also modeled the same systems and used a
multi-objective optimization model to select the technology that minimizes production costs and environmental impact.
\textcite{domingosExergyEnvironmentalAnalysis2021} evaluated the production of urea via black liquor gasification
and \textcite{gyanwaliTechnoeconomicAssessmentGreen2023} assessed the renewable urea production utilizing solid waste
and power as feedstocks.

% TODO: LITERATURE REVIEW ON SUPPLY CHAIN OPTIMIZATION
Parallel to the physico-chemical modeling of the process, modeling frameworks that incorporate optimization of
supply-chain aspects and plant location have been gaining increased attention.
\textcite{dunnettSpatiallyExplicitWholesystem2008} proposed a combined production and logistics model, formulated as a
mixed-integer linear programming (MILP) model , for the lignocellulosic ethanol production in a hypothetical geographical
area composed of a 5x5 grid and found out that the optimal production costs are highly sensitive to the spatial
distribution of biomass. \textcite{kimDesignBiomassProcessing2011} developed a MILP model for the production of bio-gasoline and
biodiesel from forestry residues that incorporates the selection of technology, capacity and biomass location into the
decision making. \textcite{schroderImprovingBiorefineryPlanning2018} implemented a combined model for a synthesis
gas biorefinery producing multiple products and used a non-linear approach coupled with an evolutionary algorithm to
determine an optimal approach in terms of location and products produced. \textcite{theozzoMILPFrameworkOptimal2021}
proposed an optimization framework for forestry products that considers not only the biomass and plant locations,
but also the storage and market layers, with his model also leveraging forestry dynamics and seasonality.

Despite these multiple works, there's still much room for improvement. While the influence of gasification parameters
on the syngas composition is well-known from gasification models and previous works, it is unclear how these
composition differences affect the performance and profitability of the urea plant as a whole, given that all studies
consider only one combination of gasifying agent and biomass. A scenario evaluation comparing the plants’ key
performance indicators (KPIs) with different biomasses and gasification conditions is therefore critical in
understanding the feasibility of renewable urea production, as is the comparison of the renewable plant with
actual process data from existing state-of-the-art urea plants.

The published process models could also be improved. Commercial urea is traditionally marketed in its granulated or prilled
form with 99 wt.\%+ purity, and all published studies deliver urea in a lower concentration, closer to 77 wt.\%,
corresponding to the concentration on the outlet of the distillation unit. The urea evaporation and condensate
treatment are energy intensive, and the recovery of ammonia dissolved in the liquid urea can improve the yield of the
process significantly \cite{meessenUreaSynthesis2014}, while other improvements such as hydrogen recovery from the
necessary ammonia purge gas are absent in recent literature. The urea process also offers interesting opportunities
in heat recovery, given that the high-pressure carbamate condenser rejects a large amount of heat at medium
temperatures, and how to maximize this heat integration is unclear since in all studies published the heat exchanger
network and utilities are abstracted from the final process model.

Finally, none of the proposed models for renewable urea production consider the biomass logistics, supply-chain and
distribution of the final product. In the case of using agricultural wastes as the feedstock, this is even more
important considering the evident synergy between receiving waste and providing fertilizer to the same locations.

\section{Methodology}

In this section a complete description of the process models, optimization model and methodology will be presented.
The process models were developed on commercial software Aspen Plus v8.8. Aspen Energy Analyzer was used to evaluate
the heat exchanger network and propose the utilities integration.

For the optimization model, a script in Python was developed, with Pyomo was chosen as the modeling language. The
commercial solver Gurobi was used to solve the model. Financial and environmental calculations were made in
Microsoft Excel.

\begin{table}

	\begin{tabular}{||c | c | >{\centering}p{1.5cm} | >{\centering}p{1.5cm} >{\centering}p{1.5cm}  c c c||}
		Name & Unit & Sugarcane bagasse & Sugarcane straw & Soybean straw & Corn & Rice & Coffee
	\end{tabular}
\end{table}



\subsection{Biomass Gasification}


\printbibliography{}


\end{document}
